%
% sa.tex
%
% (c) Roman Gassmann, HSR
%

\chapter{Einf"uhrung}
	Solange ein Optimierungsproblem gen"ugend einfach und von tiefer Dimension ist, 
	k"onnen analytische Methoden, welche zu optimalen Parameter f"uhren, angewenden werden.\\
	Dies ist aber nicht mehr der Fall wenn die Komplexit"at und die Dimension des Problems zu gross werden.
	Kommt hinzu, dass in mehr-dimensionalen und komplexeren Problemen auch h"aufig mehrere Minimas/Maximas existieren. Es muss dann eine Reihe von Tricks angewedet werden um ein einzelnes vern"unftiges/akzeptierbares Minimum/Maximum zu finden.
	
	\section{Ausgl"uhen/Abk"uhlen von Metallen}
		In gl"uhenden Metallen k"onnen sich die Atome sehr stark bewegen. Diese Bewegungsfreiheit wird mit zunehmender Abk"uhlung des Metalles verringert.
		Dabei werden sich die Atome  ordnen und schliesslich Kristalle bilden, welche eine minimale innere Energie aufweisen.\\
		Die Abk"uhlungsrate beeinflusst dabei die Bildung der Kristalle sehr stark. So f"uhrt eine zu schnelle Abk"uhlung zur Bildung von Polykristallen welche eine h"ohere innere Energie aufweisen ( Material wird br"uchig / weist spannungen auf ).\\
		
\chapter{Simuliertes Ausgl"uhen/Abk"uhlen (Simulated Annealing)}
	Die Methode des simulierten Abk"uhlens ist auf das Ausgl"uhen von gl"uhenden Metallen zur"uckzuf"uhren.
	
	%Dazu wird ein Temperatur-parameter eingef�hrt welcher der Temperatur der zu optimierenden Funktion entspricht. 
	
		